\section{文学摘抄}
%\begin{itemize}
\subsection{2020年7月28日:《破窑赋》}
%\item 2020年7月28日:《破窑赋》

%\begin{literbox}
    天有不测风云,人有旦夕祸福。蜈蚣百足,行不及蛇;雄鸡两翼,飞不过鸦。马有千里之程,无骑不能自往;人有冲天之志,非运不能自通。

    盖闻:人生在世,富贵不能淫,贫贱不能移。文章盖世,孔子厄于陈邦;武略超群,太公钓于渭水。颜渊命短,殊非凶恶之徒;盗跖年长,岂是善良之辈。尧帝明圣,却生不肖之儿;瞽叟愚顽,反生大孝之子。张良原是布衣,萧何称谓县吏。晏子身无五尺,封作齐国宰相;孔明卧居草庐,能作蜀汉军师。楚霸虽雄,败于乌江自刎;汉王虽弱,竟有万里江山。李广有射虎之威,到老无封;冯唐有乘龙之才,一生不遇。韩信未遇之时,无一日三餐,及至遇行,腰悬三齐玉印,一旦时衰,死于阴人之手。

    有先贫而后富,有老壮而少衰。满腹文章,白发竟然不中;才疏学浅,少年及第登科。深院宫娥,运退反为妓妾;风流妓女,时来配作夫人。

    青春美女,却招愚蠢之夫;俊秀郎君,反配粗丑之妇。蛟龙未遇,潜水于鱼鳖之间;君子失时,拱手于小人之下。衣服虽破,常存仪礼之容;面带忧愁,每抱怀安之量。时遭不遇,只宜安贫守份;心若不欺,必然扬眉吐气。初贫君子,天然骨骼生成;乍富小人,不脱贫寒肌体。

    天不得时,日月无光;地不得时,草木不生;水不得时,风浪不平;人不得时,利运不通。注福注禄,命里已安排定,富贵谁不欲?人若不依根基八字,岂能为卿为相?

    吾昔寓居洛阳,朝求僧餐,暮宿破窑,思衣不可遮其体,思食不可济其饥,上人憎,下人厌,人道我贱,非我不弃也。今居朝堂,官至极品,位置三公,身虽鞠躬于一人之下,而列职于千万人之上,有挞百僚之杖,有斩鄙吝之剑,思衣而有罗锦千箱,思食而有珍馐百味,出则壮士执鞭,入则佳人捧觞,上人宠,下人拥。人道我贵,非我之能也,此乃时也、运也、命也。

    嗟呼!人生在世,富贵不可尽用,贫贱不可自欺,听由天地循环,周而复始焉。
%\end{literbox}

%\end{itemize}
