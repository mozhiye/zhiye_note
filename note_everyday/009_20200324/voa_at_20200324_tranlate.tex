\\\href{https://www.51voa.com/VOA_Special_English/can-the-world-copy-south-korea-s-coronavirus-plan-84162_1.html}{Can the World Copy South Korea’s Coronavirus Plan? Translate} \newline

On Thursday morning, VOA's reporter in Seoul, William Gallo, received a text message from South Korea's emergency alert system. The message appeared on his telephone. He has received such texts hundreds of times during the coronavirus outbreak.
周四早上,美国之音驻首尔记者威廉·加洛收到了韩国紧急警报系统发来的一条短信。这条短信发到了他的手机上。在冠状病毒爆发期间,他已经收到了数百条此类短信。
Someone in Gallo's Seoul neighborhood — a 35-year-old man — had tested positive for the virus. The text message provided a link to a government website that listed everything about the man's activities for the past two days.
加洛在首尔的邻居中有一位35岁男子新冠病毒检测结果呈阳性。这条短信提供了一条指向政府网站的链接,该网站列出了该男子过去两天的所有活动信息。

The man apparently arrived at Seoul's Incheon airport at about nine in the morning. He took a train to a train station near Gallo's home, and then went to a small food store. Five hours later, he went to a restaurant. More information followed.
该男子显然是在早上约九点抵达首尔仁川机场。他坐地铁去了加洛家附近的地铁站,然后去了一家小食品店。5个小时后,他去了一家餐馆。后面还有更多信息。

By now, messages like this one are commonplace in South Korea. Gallo says his phone receives more than 10 messages about infections in his neighborhood on some days. When he goes to other parts of Seoul, his phone provides information about cases in those neighborhoods.
如今类似的短信在韩国很普遍。加洛表示,他的手机每天都收到10多条关于附近街区这几天感染病例的短信。当他去首尔其它地方时,他的手机会提供这些街区感染病例的信息。

To prepare these messages, South Korea uses in-person interviews. It also uses large amounts of personal information, bank records, phone information as well as video from cameras around the city.
为了准备这些短信,韩国采取了面谈方式。它还使用大量个人信息、银行记录、电话信息以及整个城市的摄像机拍摄的视频。

This is possible because South Korean lawmakers changed privacy laws after the outbreak of Middle East Respiratory Syndrome (MERS) in 2015. The disease caused 39 deaths in the country. Now, during dangerous outbreaks, officials can easily get everyone's private information without a court order.
这在韩国是可能实现的,因为韩国议员在2015年中东呼吸综合征爆发后修改了隐私法。这种疾病在该国导致了39人死亡。现在,在爆发危险的情况下,有关官员无需法院命令就能轻松获取每个人的隐私信息。

The plan has worked
方案奏效

As a result, South Korea has been able to identify areas with more than one coronavirus case and quickly investigate the path of the infection. It can tell those infected to stay home and warn the public to avoid that area.
结果,韩国得以确定有多起冠状病毒病例的区域,并迅速调查感染途径。它可以告诉感染者留在家里,并警告公众避开该区域。

The result has been stunning. South Korea has reported one of the lowest coronavirus death rates in the world: as of Monday, only 111 people have died out of 8,961 cases.
结果令人震惊。韩国是全球冠状病毒死亡率最低的国家之一:截至周一,该国8961起病例中只有111人死亡。

The rate of new infections has also decreased. After reaching 909 new cases a day on February 29, South Korea reported just 64 new cases on Monday.
新增感染率也有所下降。在2月29日达到一天新增909例确诊病例之后,韩国在周一仅报告了64起新病例。

South Korea's methods of fighting coronavirus have been praised as the model of how to contain the virus. It avoids forced restrictions on movement and does not lead to widespread closure of businesses.
韩国抗击冠状病毒的方法已经被誉为遏制该病毒的典范。它避免了对行动的强制性限制,并且不会导致商业大范围关闭。

Some people, however, are worried about the loss of privacy.
然而有人担心丢掉隐私。

Kenneth Roth is executive director of Human Rights Watch. He told VOA his organization is worried governments may use the threat of coronavirus to increase their powers of surveillance.
肯尼斯·罗斯是人权观察组织的执行董事。他对美国之音表示,该组织担心政府可能会利用冠状病毒的威胁来增强其监视能力。

"Once we allow them to be regularly used and give up... our right of privacy, it will be very difficult" to end it, said Roth.
罗斯表示:“一旦我们允许个人信息被定期使用,并且放弃我们的隐私权,要阻止它可就非常难了。”

If South Korea is reducing privacy in exchange for fighting the virus, many South Koreans seem to accept it happily.
如果韩国通过减少隐私权来换取抗击病毒,很多韩国人似乎很乐意接受。

Amid the coronavirus crisis, South Korean President Moon Jae-in is enjoying his highest approval ratings in months.
在冠状病毒危机中,韩国总统文在寅享有数月来的最高支持率。

In some ways, South Korea's government is helped in the fight against coronavirus by what remains of its authoritarian past, says Lee Sang-sin. He is an expert on political science and public opinion at the Korean Institute for National Unification.
李尚允表示,从某种意义上说,韩国政府的独裁历史为抗击冠状病毒提供了帮助。他是大韩民国统一研究所的政治学和舆论专家。

South Korea has a national registration system, he noted. Everyone has an identification number that must be given when buying a telephone. That has made it easier for government officials to find suspected coronavirus patients.
他指出,韩国有国家登记系统。每个人都有身份证号码,在购买电话时必须提供该号码。这让政府更容易找出冠状病毒疑似患者。

There are other reasons it may be difficult for countries to use the South Korean system to fight coronavirus.
还有其它原因让各国可能难以采用韩国抗击冠状病毒的体系。

South Korea is a small country, and is home to 51 million people. More than 50 percent live in cities and are easy to find.
韩国是一个只有5100万人口的小国家。超过50\%的人口居住在城市中,可以轻易找出来。

Most importantly, everyone in South Korea, including non-citizens, is part of a national healthcare system.
更重要的是,韩国包括非公民在内的所有人都是国家医疗保健系统的一部分。

Within the system, South Korea quickly built about 50 drive-thru testing centers. These have been praised internationally for their safety and effectiveness.
在该系统下,韩国迅速建立了约50个“得来速”检测中心。其安全性和有效性得到了全球赞誉。

Back to normal?
回归正常?

As the number of new coronavirus infections decreases, life in Seoul has begun to return to the way it was before the outbreak.
随着冠状病毒新增感染数量的减少,首尔的生活已经开始恢复到疫情爆发前的状态。

Schools are still closed, but people are out in public spaces and open areas.
学校仍然关闭,但是人们都出门走进了公共场所和空旷区域。

I'm Ashley Thompson.
我是阿什利·汤普森。(51VOA.COM原创翻译,禁止转载,违者必究!) \newline

