\href{https://www.51voa.com/VOA_Special_English/us-border-falls-quiet-with-millions-of-mexicans-barred-84264.html}{US Border Falls Quiet with Millions of Mexicans Barred} \\

By Susan Shand
05 April 2020
The world's busiest land border has fallen quiet. Restrictions meant to contain the spread of the new coronavirus have stopped millions of Mexicans from making daily trips north to the United States. That includes many who work at U.S. businesses.

At least 4 million Mexicans who live in cities along the 3,144-kilometer-long border have been affected by the restrictions on travel. The measures do not permit short crossings into American cities to visit family, get medical care or buy goods.

Many of those affected have legal "border crossing cards," which are for meant visits not related to work. Reuters reporters spoke to more than 20 people who live in Tijuana, Nogales and Ciudad Juarez. Most use their cards to care for family members on the U.S. side of the border. Some use them to work illegally.

\begin{figure}[h] %h:表示把图片放在当前位置
    \includegraphics{voa_picture.jpg}
    \caption{A police tape is seen near the beach and the Mexico-U.S. border fence, after municipal beaches are closed as part of social distancing measures to control the spread of the coronavirus in Tijuana, Mexico March 30, 2020.}
\end{figure}

All of those who spoke with the reporters said they could no longer make the crossing. This has also affected businesses on the U.S. side of the border that hire them – illegally – for agricultural jobs.

"I don't know what I'm going to do without money. I'm just waiting for a miracle," said 28-year-old Rosario Cruz. She is a mother of two young children and works for a cleaning company.

The coronavirus restrictions have stopped all non-essential travel across the border. However, the restrictions do not stop Americans from going to Mexico.

The U.S. Immigration and Customs Enforcement agency said it did not have an estimate of how many Mexicans with border crossing cards work illegally in the United States. But U.S. and Mexican immigration experts believe the number is high.

The U.S. State Department says more than 4 million border cards have been issued since 2015. The cards can be used for 10 years.

Before the coronavirus restrictions, more than 950,000 people entered the United States from Mexico every day. That information comes from the U.S. Customs and Border Protection (CBP) agency.

Andrew Selee is president of the Migration Policy Institute, based in Washington, D.C. He said limiting border crossing to fight the pandemic was understandable. But he worries that in cities such as San Diego, California, or El Paso, Texas, "businesses that really should be open in the middle of a crisis might find that they don't have employees."

"We're talking about farm work, we're talking about caregiving," he said.

In U.S. border cities like El Paso and San Diego, the effects are already being felt.

Cindy Ramos-Davidson is the chief executive of the El Paso Hispanic Chamber of Commerce. She said the lack of Mexican shoppers was "devastating" for small businesses downtown. She was also concerned about the nearby farms that use Mexican workers.

"They depend on farm workers, the day workers," she said. She noted that many of these workers use their cards to work in the U.S. illegally.

Paola Avila is a vice president of the San Diego Regional Chamber of Commerce. She said the city's so-called retail tourism has been badly affected. Retail tourism describes money spent by Mexicans who cross the border to buy certain goods.

Avila is also worried about the effect on U.S. citizens who are cared for by family members who cross from Mexico.

"If the hospitals overflow, as we predict, and they start sending people to be cared for at home, who will care for them?" she asked.

I'm John Russell.

The Reuters News Agency reported this story. Susan Shand adapted it for Learning English. Ashley Thompson was the editor.


\underline{-------------------------------------------------------------------}

Words in This Story \\
miracle - n. a wonderful event that is believed to be caused by God \\

essential - n. important, most needed \\

pandemic - n. an occurrence in which a disease spreads quickly to a large number of people around the world \\

devastate - v. to cause great harm \\

certain - adj. specific things \\

retail - adj. selling to the public in a store \\

tourism - n. the activity of going to places for pleasure \\




