\href{https://www.51voa.com/VOA_Special_English/very-and-too-84302.html}{Very and Too} \\

By Dr. Jill Robbins
10 April 2020
This week we answer a question from Rafael. He asks,

Question:

What's the difference between too and very?

Answer:

Dear Rafael,

Thanks for asking this question. These words often cause problems for people who are learning English. That is because translating them into your own language may not give you a complete understanding of how they are used.

\begin{figure}[h] %h:表示把图片放在当前位置
    \centering
    \includegraphics{voa_picture.jpg}
    \caption{Chocolates in a box}
\end{figure}

"Very" and "too" are both adverbs. They come before an adjective.

The basic difference is that "very" emphasizes the word that follows it. "Too" before a word means there is more than what is wanted. You can see how this works clearly with the adjective "much" in these sentences:

I love chocolate very much. I eat one piece of chocolate a day.

He loves chocolate too much. He eats a whole box of chocolates every day.

Very

We use "very" to show there is a higher degree of some quality. It often appears in sentences with a positive meaning:

That is a very good movie.

My dog is always very happy to see me.

Too

On the other hand, "too" means there is more of the quality than you want. This shows a negative idea. For example,

That movie is too violent for me.

What's the difference?

The main difference between "very" and "too" is that using "too" suggests that there is some problem. On a really hard day, you might come home and say:

I am too tired to eat, so we should not go out for dinner.

On a better day, you might say:

I am very tired, but I can go out for dinner.

Do you like Thai food? It has many spices. Someone who likes it would say:

I love Thai food: it is very spicy.

Someone who does not like spices would say:

Thai food is too spicy.

And that's Ask a Teacher for this week. Thank you very much for asking your question.

I'm Jill Robbins.

Dr. Jill Robbins wrote this story for Learning English. Hai Do was the editor.

\underline{------------------------------------------------------------------}

Words in This Story \\

emphasize - v. to give special importance or attention to something \\

adverb – n. a word that describes a verb, an adjective, another adverb, or a sentence and that is often used to show time, manner, place, or degree \\

positive – adj. thinking about the good qualities of someone or something \\

negative – adj. thinking about the bad qualities of someone or something \\

spicy – adj. of food: flavored with or containing strong spices and especially ones that cause a burning feeling in your mouth \\

Do you have a question for the teacher? We want to hear from you. Write to us in the Comments Section. \\

