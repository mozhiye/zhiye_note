By Anna Matteo
21 March 2020
Now, it's time for Words and Their Stories, a program from VOA Learning English.
The internet has taught us many things. One important lesson is this: Cats are funny, strange and often unpredictable animals. Their quirky personalities have made cats popular as household pets in homes across the United States.
And they are great hunters. Mice are often on their top five kill list. But sometimes, a cat will not kill the animal it catches. Instead, the catch becomes kind of a toy or plaything.
The cat seizes a mouse. Then lets it go. Catches it. Lets it go. Over and over again! It is good fun for the cat but frightening for the poor mouse.
So, if you play cat and mouse with someone, you play a power game with them. You, the person with the power, are the cat. Maybe you pretend to give the other person something they want but then take it away. You toy with them. We often call a situation like this a cat-and-mouse game. For example, a child might offer something sweet to her little brother and then take it away when he reaches for it.
A cat-and-mouse game is filled with lots of action: chases, near captures and repeated escapes. In extreme cases, a cat-and-mouse game is never-ending.
Now, let's leave the game of cat and mouse and talk about playing in a different way.
Let's imagine a classroom filled with young children. The teacher must leave the room for a couple of minutes. At first, everyone is well-behaved. But the longer the teacher is gone, the more the class acts up. It starts out slowly. Someone throws a piece of paper. A couple of students get up and walk around. Others start laughing loudly. Someone yells! Suddenly, all of the students are doing things they would never do with the teacher in the room.
But you know what we say: When the cat's away, the mice will play!
This expression means that people sometimes misbehave when no one is there to watch them. So, the teacher is the cat and the students are the mice.
We often use this expression when talking about children and parents, employees and office supervisors and even people in committed relationships.
And that brings us to the end of this Words and Their Stories.
Until next time ... I'm Anna Matteo.
Anna Matteo wrote this story for VOA Learning English. George Grow was the editor.

\underline{-------------------------------------}

Words in This Story
lesson – n. an activity that you do in order to learn something

quirky – adj. unusual in especially an interesting or appealing way

pet – n. a tame animal kept as a companion rather than for work

pretend – v. to give a false appearance of being, possessing, or performing




