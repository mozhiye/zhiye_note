\href{https://www.51voa.com/VOA_Special_English/program-collects-smartphones-for-coronavirus-hospital-patients-to-use-84557.html}{Program Collects Smartphones for Coronavirus Patients to Use} \\
 
By Ashley Thompson
17 May 2020
It is one of the many cruelties of the pandemic. To prevent the spread of the coronavirus, many people hospitalized with COVID-19 cannot have visitors. This means the patients are unable to celebrate life events with loved ones – or even say a final goodbye.

Kaya Suner came up with a solution. The 19-year-old from Rhode Island found a way to collect used smartphones and tablets and get them to patients suffering from the disease. The devices can help the patients communicate with their friends and family.

\begin{figure}[h] %h:表示把图片放在当前位置
    \centering
    \includegraphics{voa_picture.jpg}
    \caption{In this Monday, May 4, 2020, photo, Kaya Suner holds up an iPhone and an iPad as he stands for a picture in Northfield, New Hampshire, USA. (AP Photo/Charles Krupa)}
\end{figure}

His idea has started to spread.

"Kaya, you're 19, but you're a superstar, you're a hero," Rhode Island Governor, Gina Raimondo, said at a news conference in April. Raimondo's son donated an iPad to the cause.

The idea came from Suner's desire to help. He considered making protective face coverings. But his parents, both emergency room doctors, inspired him to do more.

One day, Suner was talking online with his mother, who is living separately from her son because her job puts her at a higher risk of getting sick. His mother told him that many of her patients are old and have no way to stay in touch with loved ones while they are hospitalized. Feelings of loneliness are common among the patients.

"There's no way for these sick patients to communicate with their loved ones due to the visitation ban in hospitals," Suner said. "It's really unfortunate that that's what's going on..."

So, he and a friend asked for donations of used smartphones and tablets to give to those patients.

They started with a simple request on Facebook. That effort developed into covidconnectors.org. The website lets people donate "gently used" devices that can record video.

Patients have used the donated devices for everything from celebrating birthdays and meeting new grandchildren to saying final goodbyes, Suner said.

"One family member said that they had someone in the hospital who wanted their last rites read," he said. "We were able to get an iPad to them..."

The program has been a success. In fact, the needs of Rhode Island's coronavirus patients have been met. The program is now collecting devices for medical centers in nearby Massachusetts and New Hampshire. Suner hopes to expand into New York soon.

He is concerned that as some states start to reopen, people will think there is no need for donations. But, he said, that is not the case.

"This is still a really, really large issue inside of hospitals," he said.

I'm Ashley Thompson.

The Associated Press reported this story. Ashley Thompson adapted it for VOA Learning English. Mario Ritter, Jr. was the editor.

\underline{------------------------------------------------------------------}

Words in This Story \\

pandemic –n. the fast spread of an infectious disease over a very wide geographical area \\

tablet(s) –n. a very thin computer that does not have a keyboard attached \\

inspire(d) –v. to cause someone to want to do something \\

unfortunate –adj. not a good situation, not a desirable condition \\

last rites –n. a religious ceremony that is performed by Catholic priests for someone who is dying \\
