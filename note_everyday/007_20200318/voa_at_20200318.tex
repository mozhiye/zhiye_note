By Susan Shand
16 March 2020
U.S. researchers on Monday gave the first shot to the first volunteer in a test for an experimental coronavirus vaccine. Scientists at the Kaiser Permanente Washington Health Research Institute in Seattle are carrying out the trial.

"We're team coronavirus now," Kaiser Permanente scientist Dr. Lisa Jackson said.

The Associated Press watched the first volunteer, an employee of a small technology company, receive the injection. In total, the trial study will give 45 young, healthy volunteers two doses of the vaccine, one month apart.

Public health officials say it will take 12 to 18 months for a vaccine to be approved.

The vaccine being tested in Seattle was developed by the U.S. National Institutes of Health and the biotechnology company Moderna. There is no chance the volunteers will get sickened from the vaccine; it does not contain the virus.

The goal of the trial is only to make sure the vaccine does not have any worrisome side effects. Once it is proven to be safe, larger trials will begin.

Research groups around the world are working to create a vaccine for COVID-19, the disease caused by the new coronavirus. They are creating different kinds of vaccines. Some are shots developed from new technologies. These are faster to produce than traditional shots and may be stronger. Some research groups are even aiming for temporary vaccines, which would keep people safe for a month or two while longer-lasting protection is developed.

Next month, Inovio Pharmaceuticals will begin testing the safety of its vaccine in the American states of Pennsylvania and Missouri. Similar tests will then take place in China and South Korea.

Even if early tests go well, "you're talking about a year to a year and a half" before any vaccine could be ready for widespread use, said Dr. Anthony Fauci. He is director of the NIH's National Institute of Allergy and Infectious Diseases.

Vaccine manufacturers are required to do additional studies on thousands of people to be sure a vaccine does not cause harm. But waiting is hard for the fearful public.

U.S. President Donald Trump has been pushing for a vaccine. He recently said that the work is "moving along very quickly" and that he hoped to see a vaccine "soon."

There are no proven treatments for COVID-19. In China, scientists have been experimenting with HIV drugs as well as an experimental drug named remdesivir, which was created to fight Ebola.

In the United States, the University of Nebraska Medical Center also began testing remdesivir in some Americans who were found to have COVID-19.

For most people, the new coronavirus causes mild sickness, including a fever and cough. But for older people and people with serious health problems, it can cause more severe illness, including pneumonia.

The worldwide outbreak has sickened more than 175,000 people and left more 6,000 dead.

I'm Ashley Thompson.

The Associated Press reported this story. Susan Shand adapted it for Learning English. George Grow was the editor.\newline

\underline{uuuuuuuuuuuuuuuuuuuuuuuuuuuuuuu}

Words in This Story
clinical – adj. medical or scientific work done with human beings

dose – n. a prescribed amount of medication

pneumonia – n. a disease of the lungs that makes breathing difficult

outbreak – n. the sudden occurrence of war or disease

