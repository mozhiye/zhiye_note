%==========================================
% Latex 命令环境设置
%==========================================

%==========================================
% 字体设置
%==========================================
%\newfontfamily{\monoca}{Monaco}
\newfontfamily{\monoca}{Microsoft YaHei}
% \setCJKmainfont{FandolSong}
% \setmainfont{FiraSans-Light}
% \setsansfont{FiraSans-Hair}
% \setmonofont{FiraMono-Regular}

\setCJKmainfont{Microsoft YaHei}
% \setmainfont{Microsoft YaHei}
% \setsansfont{Microsoft YaHei}
% \setmonofont{Microsoft YaHei}

\setmainfont{Consolas}
\setsansfont{Consolas}
\setmonofont{Consolas}


%==========================================
% 自定义命令设置
%==========================================
\newcommand{\urllink}[2]{\href{#1}{#2}}

\newcommand{\HT}{\textsc{\raisebox{0.1em}{H}\raisebox{-0.1em}{I}%
	\raisebox{0.1em}{T}\raisebox{-0.1em}{E}\raisebox{0.1em}{C} }}

\newtheorem{thm}{定理}
\newcommand\degree{^\circ}

\newtcbox{\emphasizebox}[1][blue]{enhanced,on line,%drop fuzzy shadow,
arc=2pt,outer arc=2pt,colback=#1!5!white,colframe=#1!50!black,
boxsep=0pt,left=3pt,right=3pt,top=1pt,bottom=1pt,
colupper=blue!50!black,fit basedim=10pt,
boxrule=0.1pt,bottomrule=0.1pt,toprule=0.1pt,nobeforeafter}

%==========================================
% 自定义lsting设置
%==========================================
\newtcblisting{messagebox}{%
breakable,
left=3mm,
listing only,
boxrule=0.2mm,
colback=gray!5,
fontupper=\monoca,colupper=red!50!black,
coltext=black
}%

\newtcblisting{commandbox}{%
breakable,
left=3mm,
listing only,
boxrule=0.2mm,
colback=black,
fontupper=\monoca,colupper=red!50!black,
coltext=green
}%

\newtcblisting{codeout}{%
breakable,
left=3mm,
boxrule=0.2mm,
colback=gray!5,
fontupper=\monoca,colupper=red!50!black,
coltext=black
}%

%\lstset{
    % numbers=left,
    % %numberstyle={\color{lightgray}},
    % numberstyle={\color{green}},
    % backgroundcolor={\color[RGB]{41, 47, 51}}, %背景颜色
    % basicstyle={\color[RGB]{208, 214, 219}}, %普通字符串颜色
    % stringstyle={\color[RGB]{0, 128, 0}}, %字符串颜色
    % keywordstyle={\color[RGB]{101, 140, 230}}, %关键词颜色
    % commentstyle={\color{gray}}, %注释颜色
    % frame=none, %无边框
    % breaklines=true, %自动分行
    % language={[ANSI]C},
    % captionpos=b,
% }

\lstnewenvironment{myccode}[1][]
{\lstset{
    numbers=left,
    %numberstyle={\color{lightgray}},
    %frame=none, %无边框
    frame=lines, %上下线
    %frame=single, %边框
    language={[ANSI]C},
    breaklines=true, %自动分行
    %keywordstyle={\color{blue}}, %关键词颜色
    %stringstyle={\color{orange}}, %字符串颜色
    %stringstyle={\color{magenta}}, %字符串颜色
    %commentstyle={\color{green}}, %注释颜色
    %basicstyle={\color[black]}, %普通字符串颜色
    %captionpos=b,
    #1
        }
}
{}

%==========================================
% 标签格式
%==========================================
\hypersetup{
    colorlinks=true,
    bookmarksnumbered=true,
    pdftitle={My LaTeX2e note},
    pdfkeywords={LaTex, note},
}

%==========================================
% 摘抄格式
%==========================================
\newenvironment{literbox}
%{\begin{messagebox}
{\begin{quote}\zihao{-4}\kaishu
%{\zihao{-3}\kaishu
}
%{\end{messagebox}}
{\end{quote}}
%{}

%==========================================
% 图片路径
%==========================================
\graphicspath{{figure/}, 
{005_protocol_note/bt_picture/}, 
{009_soft_install_note/gvim_install_picture/}, 
{008_system_work_note/system_note_picture/}, 
}

%==========================================
% 水印
%==========================================
%\usepackage{draftwatermark}
%\SetWatermarkText{Zero Note} % the Text
%\SetWatermarkLightness{0.9} % the lightness from 0 to 1, default 0.8
%\SetWatermarkScale{1.0} % the scale, default 1.2
