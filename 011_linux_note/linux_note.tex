\section{Linux系统使用笔记}
\subsection{网络传输文件}
从本地复制到远程
\begin{commandbox}
scp local_file remote_username@remote_ip:remote_folder 
\end{commandbox}
或者 
\begin{commandbox}
scp local_file remote_username@remote_ip:remote_file 
\end{commandbox}

从远程复制到本地
\begin{commandbox}
scp root@www.runoob.com:/home/root/others/music /home/space/music/1.mp3 
\end{commandbox}

\subsection{Tmux使用}
\subsubsection{Tmux一些名词约定}
\begin{itemize}
\item session:会话,通常我们在终端中操作一个任务的时候,一旦终端关闭,任务也就结束了,被强制关闭了,在 tmux 中 使用 session 就可以解决这个问题,我们可以把当前操作的任务隐藏起来,在视觉上让它消失,任务继续执行着,当我们想返回任务做一些操作的时候,它可以很方便的回来,我们通常把上面的操作就做 session 操作,我们可以把 session 给隐藏起来,我们也可以把 session 给真的关掉。
\item window:窗口
\item pane:窗格,在 tmux 中有一个窗口的概念,我们可以这样要去理解窗口:当前呈现在我们面前的这一个工作区域就是一个窗口(当前的终端界面),窗口可以被不断切割,切割成一个个小块,这一个个小块我们叫做窗格(pane),这就是窗口和窗格的概念,我们把它想象成一块大蛋糕可以切成很多小块蛋糕,窗口可以被分割成很多小的窗格。
\end{itemize}
总结:一个 session 通常指一个任务里面可以有很多窗口,一个窗口又可以有很多的窗格。可能很抽象,通过下面的实践操作,相信你会加深对 tmux 的理解。
\subsubsection{Tmux命令操作}
session操作:
\begin{itemize}
\item 新建一个session
\begin{commandbox}
± tmux
\end{commandbox}
名字默认用数字编号,指定session名
\begin{commandbox}
± tmux new -s br28_sdk
\end{commandbox}
此时从普通终端进入到seesion中。
\item 断开session
\begin{commandbox}
± tmux detach
\end{commandbox}
这时从seesion回到普通终端,新建的session还在。
\item 查看新建的session列表
\begin{commandbox}
± tmux ls    
br28_sdk: 1 windows (created Sat Sep 26 14:19:21 2020) [189x37] (attached)
\end{commandbox}
\item 进入已建立的seesion
\begin{commandbox}
± tmux attach -t br28_sdk
\end{commandbox}
\item 删除已建立的seesion
\begin{commandbox}
± tmux kill-session -t br28_sdk
\end{commandbox}
\item 在不同seesion中切换
\begin{commandbox}
± tmux switch -t br28_sdk
\end{commandbox}
\item seesion重命名
\begin{commandbox}
± tmux rename-session -t <old-session-name> <new-session-name>
\end{commandbox}
\end{itemize}

window操作:
一块工作屏幕我们叫做窗口(window),窗口是可以被分割的,当前的工作区域被分割的一块块区域就是窗格(pane)。
\emphasizebox{窗口内部操作}
\begin{itemize}
\item 水平切割窗格
\begin{commandbox}
± tmux split-window 
\end{commandbox}
垂直切割窗格
\begin{commandbox}
± tmux split-window -h
\end{commandbox}

\item 在不同窗格中移动, 向上移动
\begin{commandbox}
± tmux select-pane  -U
\end{commandbox}
向下移动
\begin{commandbox}
± tmux select-pane  -D
\end{commandbox}
向左移动
\begin{commandbox}
± tmux select-pane  -L
\end{commandbox}
向右移动
\begin{commandbox}
± tmux select-pane  -R
\end{commandbox}

\emphasizebox{窗口间操作}

\item 创建一个window
\begin{commandbox}
± tmux new-window -n <window-name>
\end{commandbox}

\item 切换window
\begin{commandbox}
± tmux select-window -t <window-name>
\end{commandbox}

\item 重命名window
\begin{commandbox}
± tmux rename-window <new-window-name>
\end{commandbox}

\item 关闭window
\begin{commandbox}
± tmux kill-window -t <window-name>
\end{commandbox}

\end{itemize}

\begin{itemize}
\item 列出所有快捷键,及其对应的 Tmux 命令
\begin{commandbox}
 tmux list-keys
\end{commandbox}

\item 列出所有 Tmux 命令及其参数
\begin{commandbox}
± tmux list-commands
\end{commandbox}

\item 列出当前所有 Tmux 会话的信息
\begin{commandbox}
± tmux info
\end{commandbox}

\item 重新加载当前的 Tmux 配置
\begin{commandbox}
± tmux source-file ~/.tmux.conf
\end{commandbox}

\end{itemize}

\subsubsection{Tmux配置文件}
在根目录新建一个.tmux.conf文件
\begin{messagebox}
unbind C-b
set -g prefix C-a

bind-key -n F11 previous-window
bind-key -n F12 next-window

unbind %
bind | split-window -h

unbind '"'
bind - split-window -v
#bind a rekiad key
# bind r source-file ~/.tmux.conf ; display-message "Config reloaded.."

#选择分割的窗格
bind k selectp -U #选择上窗格
bind j selectp -D #选择下窗格
bind h selectp -L #选择左窗格
bind l selectp -R #选择右窗格
#----------------------------------------------


#重新调整窗格的大小
bind C-k resizep -U 5
bind C-j resizep -D 5
bind C-h resizep -L 5
bind C-l resizep -R 5
#----------------------------------------------


#use vim keybindings in copy mode
# setw -g mode-keys vi

#status bar
#color
set -g status-bg black
set -g status-fg white

#align
set-option -g status-justify centre

# left conner
set-option -g status-left '#[bg=black,fg=green,bright][#[fg=cyan]#S#[fg=green]]'
set-option -g status-left '#[bg=black,fg=green,bright][#[fg=cyan]#S#[fg=green]]'
set-option -g status-left-length 20

# window list
setw -g automatic-rename on
set-window-option -g window-status-format '#[dim]#I:#[default]#W#[fg=grey,dim]'
set-window-option -g window-status-current-format '#[fg=cyan,bold]#I#[fg=blue]:#[fg=cyan]#W#[fg=dim]'

# right conner
set -g status-right '#[fg=green][#[fg=cyan]%Y-%m-%d#[fg=green]]'
set -g status-right "#[fg=yellow,bright][ #[fg=cyan]#W #[fg=yellow]]#[default] #[fg=yellow,bright]- %Y.%m.%d #[fg=green]%H:%M #[default]"

set-window-option -g mode-keys vi #可以设置为vi或emacs
set-window-option -g utf8 on #开启窗口的UTF-8支持

# ---持久保存Tmux会话
# for resurrect
#run-shell ~/.tmux/plugins/tmux-resurrect/resurrect.tmux
# for continuum
#run-shell ~/.tmux/plugins/tmux-continuum/continuum.tmux
#5分钟自动保存一次
#set -g @continuum-save-interval '5' 

# ---tmux powerline
#set-option -g status on
#set-option -g status-interval 2
#set-option -g status-utf8 on
#set-option -g status-justify "centre"
#set-option -g status-left-length 60
#set-option -g status-right-length 90
#set-option -g status-left "#(~/.tmux/plugins/tmux-powerline/powerline.sh left)"
#set-option -g status-right "#(~/.tmux/plugins/tmux-powerline/powerline.sh right)"
#set-window-option -g window-status-current-format "#[fg=colour235, bg=colour27]⮀#[fg=colour255, bg=colour27] #I ⮁ #W #[fg=colour27, bg=colour235]⮀"
set -g default-terminal "screen-256color"

# ---Tmux Plugin Manager
# List of plugins
set -g @plugin 'tmux-plugins/tpm'
set -g @plugin 'tmux-plugins/tmux-sensible'
set -g @plugin 'erikw/tmux-powerline'
#set -g @plugin 'tmux-colors-solarized'

#set -g @colors-solarized 'light'

# Initialize TMUX plugin manager (keep this line at the very bottom of tmux.conf)
run '~/.tmux/plugins/tpm/tpm'
\end{messagebox}

\subsubsection{Tmux快捷键}
会话快捷键
\begin{itemize}
\item 前缀 + ?: 帮助
\item 前缀 + ?: 帮助
\item 前缀 + d:分离当前会话
\item 前缀 + s:列出所有会话。
\item 前缀 + \$:重命名当前会话
\end{itemize}

窗格快捷键
\begin{itemize}
\item 前缀 +  \%:划分左右两个窗格。
\item 前缀 +  ":划分上下两个窗格。
\item 前缀 +  <arrow key>:光标切换到其他窗格。<arrow key>是指向要切换到的窗格的方向键,比如切换到下方窗格,就按方向键↓。
\item 前缀 +  ;:光标切换到上一个窗格。
\item 前缀 +  o:光标切换到下一个窗格。
\item 前缀 +  {:当前窗格与上一个窗格交换位置。
\item 前缀 +  }:当前窗格与下一个窗格交换位置。
\item 前缀 +  Ctrl+o:所有窗格向前移动一个位置,第一个窗格变成最后一个窗格。
\item 前缀 +  Alt+o:所有窗格向后移动一个位置,最后一个窗格变成第一个窗格。
\item 前缀 +  x:关闭当前窗格。
\item 前缀 +  !:将当前窗格拆分为一个独立窗口。
\item 前缀 +  z:当前窗格全屏显示,再使用一次会变回原来大小。
\item 前缀 +  Ctrl+<arrow key>:按箭头方向调整窗格大小。
\item 前缀 +  q:显示窗格编号。
\end{itemize}

窗口快捷键
\begin{itemize}
\item 前缀 + c:创建一个新窗口,状态栏会显示多个窗口的信息。
\item 前缀 + p:切换到上一个窗口(按照状态栏上的顺序)。
\item 前缀 + n:切换到下一个窗口。
\item 前缀 + <number>:切换到指定编号的窗口,其中的<number>是状态栏上的窗口编号。
\item 前缀 + w:从列表中选择窗口。
\item 前缀 + ,:窗口重命名。
\end{itemize}

时间快捷键
\begin{itemize}
\item 前缀 + t:在当前的窗格当中显示时钟,非常酷炫的一个功能,点击 enter (回车键将会复原)
\end{itemize}

